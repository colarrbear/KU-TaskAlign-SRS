\chapter{Introduction}
\label{chap:introduction}

\section{Background}
\label{section:background}

Many conventional electronic equipment factories in Thailand still rely on manual scheduling, where production tasks are planned based on human experience rather than optimized strategies. This traditional approach often leads to inefficiencies in resource utilization and production planning. As industries move toward digital transformation, electronic factories face challenges in modernizing their workflows.
Our project, TaskAlign, seeks to address these challenges by implementing a job scheduling system that optimizes user-specified criteria, such as production time, machine utilization, and other relevant factors. By incorporating detailed machine-specific data and product specifications, including manufacturing steps and required components, the system generates the most efficient production plan based on the specified parameters. This approach ensures the optimal utilization of resources in the manufacturing process.

\section{Problem Statement}
\label{section:problem-statement}

Designing the production scheduling in traditional factories with non-digital machinery usually relies on the experience of the managers rather than automatic optimized scheduling strategies. Without data-driven decision-making, task allocation may not be efficient, leading to extended machine idle times, production bottlenecks, and suboptimal workforce utilization. These inefficiencies increase operational costs and reduce overall productivity.
This challenge becomes even more critical as factories scale operations or introduce new product lines, where traditional experience-based scheduling struggles to adapt to complexity. Our project, TaskAlign, provides a solution by allowing factories to input production data into a web application. The system subsequently optimizes job scheduling based on the user's specified criteria, taking into account machine specifications and product assembly requirements ensuring improved efficiency without requiring costly equipment upgrades.

\section{Solution Overview}
\label{section:solution-overview}

TaskAlign is a task scheduling system designed for conventional electronic equipment factories that rely on manual production management. The system assists factory managers and operators in designing job schedules that optimize user-specified optimization criteria, such as production time, machine utilization and workforce allocation, while taking into account constraints like machine cool-down time, energy consumption, and production dependencies. A key feature of TaskAlign is its data encoding system, which transforms user inputs (such as machine specifications, worker assignments, product requirements, and dependencies) into a representation of the factory's workflow that is suitable for job scheduling algorithms. The system then applies optimization algorithms to generate an optimal production schedule that maximizes machine utilization and efficient workforce allocation while respecting constraints like machine cool-down time and production dependencies. Users can view and interact with the resulting schedules through Gantt charts and timeline visualizations, allowing them to make manual adjustments to production plans when necessary.

\subsection{Features}
\label{subsection:features}

\begin{enumerate}[leftmargin=80pt]
    \item \textbf{Smart Task Scheduling (Optimization-Based)}
    \begin{itemize}
        \item Users input machine data, including:
        \begin{itemize}
            \item Machine name
            \item Machine type
            \item Machine capabilities
            \item Number of fixed workers assigned to operate the machine
            \item Maximum operation time
            \item Cooldown time
            \item Defect rate
            \item Production rate
            \item Status of machine (Active, Maintenance, Inactive)
        \end{itemize}
        \begin{itemize}
            \item General product details (Name, Model, Description)
            \item Product components:
            \begin{itemize}
                \item ID (Might be auto generated)
                \item Component name
                \item Component steps: Steps in assembling this components (Additional details can be added in the "Assembly Process")
                \item Total Production time per component production time
                \item Required machines
                \item Prerequisite components
                \item Component quantity
            \end{itemize}
            \item Machine capabilities
            \item Number of fixed workers assigned to operate the machine
            \item Maximum operation time
            \item Cooldown time
            \item Defect rate
            \item Production rate
            \item Status of machine (Active, Maintenance, Inactive)
        \end{itemize}
        \item The system generates an optimal production schedule considering constraints like machine efficiency, energy usage, and human resource availability.
    \end{itemize}

    \item \textbf{Flexible Scheduling Modes}  
    \begin{itemize}
        \item \textbf{Fresh Start Mode:} Creates a new schedule for factories starting production from zero.  
        \item \textbf{Production Resume Mode:} Users can input completed portions, and the system will recalculate the optimal production plan to complete the remaining work.  
    \end{itemize}

    \item \textbf{Visualized Task Planning (Gantt Chart \& Timeline View)}
    \begin{itemize}
        \item Displays a clear production roadmap, helping managers track daily tasks and machine utilization.
        \item Allows users to adjust schedules manually if needed.
    \end{itemize}
\end{enumerate}

\subsection{Optional Features}
\label{subsection:optional features}
\begin{enumerate}[leftmargin=80pt]
    \item \textbf{Cost \& Resource Optimization}
    \begin{itemize}
        \item Users can input energy consumption, carbon emissions, human cost, etc.
        \item The system analyzes cost-effectiveness and suggests ways to reduce waste and optimize efficiency.
    \end{itemize}
\end{enumerate}



\section{Target User}
\label{section:target-user}

\subsection{Target Audience and Challenges}

TaskAlign is specifically designed for conventional electronic equipment factories in Thailand facing production scheduling challenges. These facilities typically operate with traditional manufacturing processes and equipment that lack integrated digital systems or sensors for automated tracking. The target users are electronics factories that produce various appliances such as refrigerators, air conditioners, washing machines, and small household electronic devices across multiple production lines.

\subsubsection{Common Challenges}
These manufacturing facilities often face common challenges:
\begin{itemize}
    \item They operate with capable but aging machinery without built-in digital interfaces.
    \item Production planning relies heavily on manual methods and staff experience.
    \item They lack real-time data collection capabilities due to the absence of sensors.
    \item They need to manage multiple assembly lines producing different products simultaneously.
    \item They require flexible scheduling to accommodate seasonal demand fluctuations.
\end{itemize}

\subsubsection{Key Stakeholders}
The system serves several key stakeholders:
\begin{itemize}
    \item \textbf{Factory Managers} who oversee production operations and seek more efficient methods for scheduling tasks and allocating resources. They need comprehensive views of production timelines and resource utilization.
    \item \textbf{Operators and Technicians} who directly handle machinery and require a user-friendly interface to understand and follow optimized schedules. Since the factories lack automated tracking systems, these users need clear visual representations of the scheduling plan.
    \item \textbf{Factory Owners} looking for cost-effective digital transformation solutions that improve productivity without replacing existing machines. They seek ROI through better resource allocation rather than capital-intensive equipment upgrades.
\end{itemize}

Users typically possess substantial experience in factory operations with varying levels of technical proficiency. TaskAlign accommodates this diversity through an intuitive interface that provides both basic operational guidance and advanced optimization insights, making digital transformation accessible even to facilities with limited technology infrastructure.

\section{Benefit}
\label{section:benefit}

TaskAlign offers practical operational improvements for electronics factories. The solution optimizes production schedules according to user-specified criteria, which may include minimizing overall production time, reducing operational costs, or improving energy efficiency depending on the user's criteria. By coordinating machine operations in a logical sequence and efficiently allocating workforce, the system ensures better utilization of existing equipment.
The flexibility of TaskAlign allows factories to choose which factors to prioritize in their production planning, adapting to changing business needs or seasonal requirements. 
Most importantly, TaskAlign enables digital transformation through better process optimization rather than equipment replacement, making production improvements accessible and cost-effective for traditional factories that aren't ready for comprehensive automation upgrades.

\section{Terminology}
\label{section:terminology}

\subsection{Optimization Algorithm}
A computational method used to find the most efficient way to \textbf{schedule tasks, allocate resources, and reduce downtime} in the factory based on input constraints (e.g., machine limits, production steps).  

\subsection{Task Scheduling}
The process of \textbf{assigning tasks to machines and workers} in an optimal order, ensuring efficient production while considering machine availability, processing time, and dependencies.

\subsection{Gantt Chart}
A visual representation of the production schedule, showing \textbf{when each machine is used, what tasks are performed, and how long each step takes}.

\subsection{MRP (I)} Material requirements planning (MRP) is a system for calculating the materials and components needed to manufacture a product. It consists of three primary steps: taking inventory of the materials and components on hand, identifying which additional ones are needed and then scheduling their production or purchase. \cite{techtarget_mrp}

\subsection{MRP (II)} Manufacturing resource planning (MRP II) is an integrated information system used by businesses. It evolved from earlier materials requirement planning (MRP) systems by including the integration of additional data, such as employee and financial needs. The system is designed to centralize, integrate, and process information for effective decision-making in scheduling, design engineering, inventory management, and cost control in manufacturing. \cite{investopedia_mrpii}

\subsection{ERP} Enterprise resource planning (ERP) is a software system that helps organizations streamline their core business processes—including finance, HR, manufacturing, supply chain, sales, and procurement—with a unified view of activity and provides a single source of truth. \cite{sap_erp}

\subsection{MES} A manufacturing execution system, or MES, is a comprehensive, dynamic software system that monitors, tracks, documents, and controls the process of manufacturing goods—from raw materials to finished products. Providing a functional layer between enterprise resource planning (ERP) and process control systems, an MES gives decision-makers the data they need to make their plant floor more efficient. \cite{sap_mes}