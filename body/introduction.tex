\chapter{Introduction}
\label{chap:introduction}

\section{Background}
\label{section:background}

Many electronic factories (e.g., Thai City Electric) still rely on manual scheduling, where production tasks are planned based on human experience rather than optimized strategies. This traditional approach often leads to inefficiencies in resource utilization and production planning. As industries move toward digital transformation, electronic factories face challenges in modernizing their workflows.
Our project, TaskAlign, addresses these challenges by providing a task scheduling system that optimizes machine usage and workforce allocation. By incorporating machine data and product specifications such as manufacturing steps and required components then our system generates the most efficient production plan, ensuring that machines are utilized effectively based on the products being manufactured.

\section{Problem Statement}
\label{section:problem-statement}

Factory production planning often relies on the experience of managers rather than optimized scheduling strategies. Without data-driven decision-making, task allocation may not be efficient, leading to extended machine idle times, production bottlenecks, and suboptimal workforce utilization. These inefficiencies increase operational costs and reduce overall productivity.
This challenge becomes even more critical as factories scale operations or introduce new product lines, where traditional experience-based scheduling struggles to adapt to complexity. Our project, TaskAlign, provides a solution by allowing factories to input production data into a web application. The system then optimizes task scheduling based on machine capabilities and product requirements, ensuring improved efficiency without requiring costly equipment upgrades.

\section{Solution Overview}
\label{section:solution-overview}

TaskAlign is a task scheduling system designed for electronic factories that rely on manual production management. The system helps factory managers and operators optimize machine utilization, workforce allocation, and production planning while considering constraints like machine cooldown time, energy consumption, and production dependencies.
A key feature of TaskAlign is its advanced data encoding system, which transforms user inputs into a representation of the factory's workflow. This visual model allows for comprehensive analysis of production dependencies and constraints. The system then applies optimization algorithms to generate an optimal production schedule, which can be visualized using Gantt charts or other intuitive formats, allowing users to adjust production plans based on constraints.

\subsection{Features}
\label{subsection:features}

\begin{enumerate}[leftmargin=80pt]
    \item \textbf{Smart Task Scheduling (Optimization-Based)}
    \begin{itemize}
        \item Users input machine data (capabilities, max operation time, cooldown time, defect rate, error probability).
        \item Users define product data (required parts, assembly steps, dependencies).
        \item The system generates an optimal production schedule considering constraints like machine efficiency, energy usage, and human resource availability.
    \end{itemize}

    \item \textbf{Flexible Scheduling Modes}
    \begin{itemize}
        \item \textbf{Fresh Start Mode}: Creates a new schedule for factories starting production from zero.
        \item \textbf{Ongoing Production Mode}: Users input current production status, and the system recalculates the best way to continue.
    \end{itemize}

    \item \textbf{Visualized Task Planning (Gantt Chart \& Timeline View)}
    \begin{itemize}
        \item Displays a clear production roadmap, helping managers track daily tasks and machine utilization.
        \item Allows users to adjust schedules manually if needed.
    \end{itemize}

    \item \textbf{Cost \& Resource Optimization (Optional Advanced Mode)}
    \begin{itemize}
        \item Users can input energy consumption, carbon emissions, human cost, etc.
        \item The system analyzes cost-effectiveness and suggests ways to reduce waste and optimize efficiency.
    \end{itemize}

\end{enumerate}

\section{Target User}
\label{section:target-user}

TaskAlign is specifically designed for THAI CITY ELECTRIC COMPANY and similar electronics manufacturers facing production scheduling challenges. The company has capable manufacturing resources but needs a systematic approach to optimize their workflow sequencing and resource allocation.
The system serves:

Factory Managers who oversee production operations and seek more efficient methods for scheduling tasks and allocating resources.
Operators and Technicians who directly handle machinery and require a user-friendly interface to track and follow optimized schedules.
Factory Owners looking for cost-effective digital transformation solutions that improve productivity without replacing existing machines.

Users typically possess substantial experience in factory operations with varying levels of technical proficiency. TaskAlign accommodates this diversity through an intuitive interface that provides both basic operational guidance and advanced optimization insights.

\section{Benefit}
\label{section:benefit}

TaskAlign offers practical operational improvements for electronics factories. The solution reduces downtime by coordinating machine operations in a logical sequence, ensuring better utilization of existing equipment. By minimizing idle time and optimizing resource allocation, TaskAlign creates a more efficient production process. The system also provides flexibility for scaling operations, allowing facilities to integrate new product lines without disrupting existing workflows. Most importantly, TaskAlign enables digital transformation through better process optimization rather than equipment replacement, making production improvements accessible and cost-effective.

\section{Terminology}
\label{section:terminology}

\subsection{Optimization Algorithm}
A computational method used to find the most efficient way to \textbf{schedule tasks, allocate resources, and reduce downtime} in the factory based on input constraints (e.g., machine limits, production steps).  

\subsection{Task Scheduling}
The process of \textbf{assigning tasks to machines and workers} in an optimal order, ensuring efficient production while considering machine availability, processing time, and dependencies.

\subsection{Gantt Chart}
A visual representation of the production schedule, showing \textbf{when each machine is used, what tasks are performed, and how long each step takes}.
