\chapter{Requirement Analysis}
\label{chap:requirement-analysis}

\section{Stakeholder Analysis}
\label{section:stakeholder-analysis}

% \section*{Key Stakeholders and Their Roles}

\textbf{Factory Managers} – Oversee production operations and seek efficient scheduling methods to reduce downtime and optimize machine utilization.

\textbf{Production Planners} – Responsible for task scheduling and workforce allocation, ensuring production efficiency and minimizing bottlenecks.

\textbf{Machine Operators \& Technicians} – Directly interact with machinery and rely on TaskAlign for clear, optimized schedules to streamline manual operations.

\textbf{Factory Owners} – Interested in cost-effective solutions that improve productivity without requiring expensive equipment upgrades.

\textbf{IT \& System Administrators} – Manage the implementation and integration of TaskAlign with existing factory systems, including ERP solutions.

\textbf{Supply Chain \& Logistics Teams} – Depend on optimized production schedules for better coordination of raw material procurement and order fulfillment.

\textbf{Customers \& Clients} – Indirect stakeholders who benefit from reduced lead times, improved production efficiency, and consistent product availability.

TaskAlign addresses the needs of these stakeholders by providing an optimized, data-driven scheduling system tailored to solve small to medium electronic factories that are still reliant on manual scheduling in manufacturing environments.


\section{User Stories}
\label{section:user-stories}

% //#TODO: place table here
\begin{figure}[h]
    \centering
    \includegraphics[width=0.5\textwidth]{examples/user_story.png}
    \caption{User Story Table} 
\end{figure}

\section{Use Case Diagram}
\label{section:use-case-diagram}
<TIP: Write a use case diagram for your project here. Refer to an
article “What is a use case diagram?” by Lucidchart for help./>

\section{Use Case Model}
\label{section:use-case-model}
A use case is a detailed description of how a system
interacts with an external entity (such as a user or another system) to
accomplish a specific goal. Use cases provide a high-level view of the
functionality of a system and help in capturing and documenting its
requirements from the perspective of end users.

<TIP: Write use cases for your project here. Make sure to use the
appropriate type of use case for each scenario (brief, casual, and fully-dressed
use case)./>

\section{User Interface Design}
\label{section:user-interface-design}
<TIP: Put the initial design of your application here. You can
showcase a detailed design of a specific page or a sitemap of your application.
See an example below./>

\begin{figure}[h]
    \centering
    \includegraphics[width=0.8\textwidth]{examples/user-interface-design.png}
    \caption{User Interface Design}
\end{figure}